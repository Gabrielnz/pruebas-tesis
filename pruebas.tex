\documentclass[10pt, a4paper]{article}
\usepackage[top=3cm, bottom=3cm, left=3cm,right=2cm]{geometry}
\usepackage{setspace} %interlineado
\setlength{\parindent}{7mm} % sangria
\setlength{\parskip}{\baselineskip} %espacio entre parrafos
\usepackage{microtype} %micro-arreglos en los parrafos para que queden mejor justificados
\usepackage{placeins} % ayuda a que las tablas no queden en medio de los textos
\usepackage{tabulary} % ajusta el alto de la tabla
\usepackage{tabularx} % ajusta el ancho de la tabla
\usepackage[font={footnotesize}]{caption} % tamano para la descripcion de las tablas
\usepackage{array}
\usepackage{float}
\usepackage{makecell}
\usepackage{graphicx}
\usepackage[spanish,es-tabla,es-lcroman]{babel}
\selectlanguage{spanish}
\usepackage[authoryear,datebegin]{flexbib}
\bibliographystyle{flexbib}
\usepackage{fancyhdr}
%\pagestyle{fancy}
\fancyhead{}
%\fancyfoot{}
\fancyhead[L]{\nouppercase{\leftmark}}
%\fancyfoot[R]{\thepage}
\usepackage{gensymb}
\usepackage{amssymb}
\usepackage{blindtext}
\usepackage{amsmath}
\usepackage{listings}
\usepackage{color}
\usepackage{textcomp}
\usepackage{caption}
\usepackage{subcaption}
\usepackage{gensymb}

\newcommand{\anchotabla}{14cm}
\newcommand{\altocelda}{2pt}

\newcommand{\proyecto}{
	Software para el MiniScan XE Plus.
}

\newcommand{\nombre}{
	Gabriel N\'{u}\~{n}ez.
}

\newcommand{\fecha}{
	15 de octubre, 2015.
}

\begin{document}
	%\tableofcontents
	\spacing{1.5} % interlineado
	\begin{center}
	\textbf{Gui\'{o}n de prueba 1}
\end{center}

\textbf{Nombre del proyecto:} \proyecto

\textbf{\'{A}rea funcional:} Operaci\'{o}n del MiniScan XE Plus.

\textbf{Nombre de la prueba:} Operar el MiniScan XE Plus.
\vfill
\textbf{Prop\'{o}sito:}
\begin{table}[h]
	\centering
	\setlength{\extrarowheight}{\altocelda}
	\begin{tabularx}{\anchotabla}{|X|}
		\hline
		Probar la funcionalidad que implica operar el MiniScan XE Plus.\\ \hline
	\end{tabularx}
\end{table}

\textbf{Resultado:}
\begin{table}[h]
	\centering
	\setlength{\extrarowheight}{\altocelda}
	\begin{tabularx}{\anchotabla}{|X|}
		\hline
		Prueba superada con \'{e}xito.\\ \hline
	\end{tabularx}
\end{table}

\begin{table}[h]
		\centering
		\setlength{\extrarowheight}{\altocelda}
		\begin{tabulary}{\anchotabla}{|c|J|J|J|}
			\hline
			\thead{\textbf{\small{\#}}} & \thead{\textbf{\small{Acci\'{o}n}}} & \thead{\textbf{\small{Opci\'{o}n}}} & \thead{\textbf{\small{Resultado esperado}}}\\ \hline

			1 & El usuario conecta el software con el MiniScan. & Se selecciona la opci\'{o}n <<Conectar>>, ubicada en el men\'{u} MiniScan. & Ventana de confirmaci\'{o}n, informando al usuario que la conexi\'{o}n ha sido establecida.\\ \hline
			
			2 & El usuario realiza una medici\'{o}n. & Se selecciona la opci\'{o}n <<Realizar medici\'{o}n>>, ubicada en el men\'{u} MiniScan. & Tabla de datos espectrales recuperados de la medici\'{o}n.\\ \hline
		
			3 & El usuario desconecta el software con el MiniScan. & Se selecciona la opci\'{o}n <<Desconectar>>, ubicada en el men\'{u} MiniScan. & Ventana de confirmaci\'{o}n, informando al usuario que la desconexi\'{o}n ha sido realizada.\\ \hline
			
			4 & El usuario calibra el MiniScan. & Se selecciona la opci\'{o}n <<Estandarizar>>, ubicada en el men\'{u} MiniScan. & Ventanas de calibraci\'{o}n para la trampa negra y la cer\'{a}mica blanca del MiniScan.\\ \hline
		\end{tabulary}
\end{table}

\textbf{Errores encontrados:}
\begin{table}[H]
	\centering
	\setlength{\extrarowheight}{\altocelda}
	\begin{tabularx}{\anchotabla}{|X|}
		\hline
		\thead{\textbf{\small{Descripci\'{o}n del error}}}
		\\ \hline
		No se encontr\'{o} ning\'{u}n error.\\ \hline
	\end{tabularx}
\end{table}

\textbf{Comentarios y observaciones:}
\begin{table}[H]
	\centering
	\setlength{\extrarowheight}{\altocelda}
	\begin{tabularx}{\anchotabla}{|X|}
		\hline
		Se verific\'{o} la funcionalidad desconectando f\'{i}sicamente el MiniScan, en cuyo caso el software genera un mensaje de error al usuario.\\ \hline
	\end{tabularx}
\end{table}

\begin{minipage}[t]{0.45\textwidth}
	\begin{flushleft}
		\textbf{Elaborado por:} \nombre
	\end{flushleft}
\end{minipage}
\begin{minipage}[t]{0.45\textwidth}
	\begin{flushright}
		\begin{center}
			\textbf{Fecha:} \fecha
		\end{center}
	\end{flushright}
\end{minipage}
\vfill
\newpage
\begin{center}
	\textbf{Gui\'{o}n de prueba 2}
\end{center}

\textbf{Nombre del proyecto:} \proyecto

\textbf{\'{A}rea funcional:} Resultados.

\textbf{Nombre de la prueba:} Gestionar resultados.
\vfill
\textbf{Prop\'{o}sito:}
\begin{table}[h]
	\centering
	\setlength{\extrarowheight}{\altocelda}
	\begin{tabularx}{\anchotabla}{|X|}
		\hline
		Probar la funcionalidad que implica gestionar los resultados de las mediciones.\\ \hline
	\end{tabularx}
\end{table}

\textbf{Resultado:}
\begin{table}[h]
	\centering
	\setlength{\extrarowheight}{\altocelda}
	\begin{tabularx}{\anchotabla}{|X|}
		\hline
		Prueba superada con \'{e}xito.\\ \hline
	\end{tabularx}
\end{table}

\begin{table}[h]
		\centering
		\setlength{\extrarowheight}{\altocelda}
		\begin{tabulary}{\anchotabla}{|c|J|J|J|}
			\hline
			\thead{\textbf{\small{\#}}} & \thead{\textbf{\small{Acci\'{o}n}}} & \thead{\textbf{\small{Opci\'{o}n}}} & \thead{\textbf{\small{Resultado esperado}}}\\ \hline

			1 & El usuario consulta los resultados de una medici\'{o}n. & Se selecciona los resultados disponibles en el men\'{u} Resultados. & Ventanas con los resultados adicionales de una medici\'{o}n.\\ \hline
		
			2 & El usuario borra los resultados de una medici\'{o}n. & Se selecciona la opci\'{o}n <<Borrar resultados>>, ubicada en el men\'{u} resultados. & Ventana de confirmaci\'{o}n, informando al usuario que se han borrado los resultados correctamente.\\ \hline
		\end{tabulary}
\end{table}

\textbf{Errores encontrados:}
\begin{table}[H]
	\centering
	\setlength{\extrarowheight}{\altocelda}
	\begin{tabularx}{\anchotabla}{|X|}
		\hline
		\thead{\textbf{\small{Descripci\'{o}n del error}}}
		\\ \hline
		No se encontr\'{o} ning\'{u}n error.\\ \hline
	\end{tabularx}
\end{table}

\textbf{Comentarios y observaciones:}
\begin{table}[H]
	\centering
	\setlength{\extrarowheight}{\altocelda}
	\begin{tabularx}{\anchotabla}{|X|}
		\hline
		No aplica.\\ \hline
	\end{tabularx}
\end{table}

\begin{minipage}[t]{0.45\textwidth}
	\begin{flushleft}
		\textbf{Elaborado por:} \nombre
	\end{flushleft}
\end{minipage}
\begin{minipage}[t]{0.45\textwidth}
	\begin{flushright}
		\begin{center}
			\textbf{Fecha:} \fecha
		\end{center}
	\end{flushright}
\end{minipage}
\vfill
\newpage
\begin{center}
	\textbf{Gui\'{o}n de prueba 3}
\end{center}

\textbf{Nombre del proyecto:} \proyecto

\textbf{\'{A}rea funcional:} Sesi\'{o}n de usuarios.

\textbf{Nombre de la prueba:} Gestionar sesi\'{o}n de usuarios.
\vfill
\textbf{Prop\'{o}sito:}
\begin{table}[h]
	\centering
	\setlength{\extrarowheight}{\altocelda}
	\begin{tabularx}{\anchotabla}{|X|}
		\hline
		Probar la funcionalidad que implica gestionar la sesi\'{o}n de un usuario.\\ \hline
	\end{tabularx}
\end{table}

\textbf{Resultado:}
\begin{table}[h]
	\centering
	\setlength{\extrarowheight}{\altocelda}
	\begin{tabularx}{\anchotabla}{|X|}
		\hline
		Prueba superada con \'{e}xito.\\ \hline
	\end{tabularx}
\end{table}

\begin{table}[h]
		\centering
		\setlength{\extrarowheight}{\altocelda}
		\begin{tabulary}{\anchotabla}{|c|J|J|J|}
			\hline
			\thead{\textbf{\small{\#}}} & \thead{\textbf{\small{Acci\'{o}n}}} & \thead{\textbf{\small{Opci\'{o}n}}} & \thead{\textbf{\small{Resultado esperado}}}\\ \hline

			1 & El usuario inicia su sesi\'{o}n, suministrando su c\'{e}dula de identidad y su contrase\~{n}a. & Se selecciona la opci\'{o}n <<Iniciar sesi\'{o}n>>, ubicada en el men\'{u} usuario. & Ventana de confirmaci\'{o}n, informando al usuario que ha iniciado sesi\'{o}n correctamente.\\ \hline
		
			2 & El usuario ve sus datos personales. & Se selecciona la opci\'{o}n <<Ver usuario>>, ubicada en el men\'{u} usuario. & Ventana de muestra los datos personales del usuario.\\ \hline
			
			3 & El usuario modifica sus datos personales, proporcionando su contrase\~{n}a. & Se selecciona la opci\'{o}n <<Modificar usuario>>, ubicada en el men\'{u} usuario. & Ventana de confirmaci\'{o}n, informando al usuario que ha modificado sus datos correctamente.\\ \hline
			
			4 & El usuario cambia su contrase\~{n}a, proporcionando su contrase\~{n}a actual y su contrase\~{n}a nueva. & Se selecciona la opci\'{o}n <<Cambiar contrase\~{n}a>>, ubicada en el men\'{u} usuario. & Ventana de confirmaci\'{o}n, informando al usuario que ha cambiado su contrase\~{n}a correctamente.\\ \hline
			
			5 & El usuario cierra su sesi\'{o}n. & Se selecciona la opci\'{o}n <<Cerrar sesi\'{o}n>>, ubicada en el men\'{u} usuario. & Ventana de confirmaci\'{o}n, informando al usuario que ha cerrado su sesi\'{o}n correctamente.\\ \hline
		\end{tabulary}
\end{table}

\textbf{Errores encontrados:}
\begin{table}[H]
	\centering
	\setlength{\extrarowheight}{\altocelda}
	\begin{tabularx}{\anchotabla}{|X|}
		\hline
		\thead{\textbf{\small{Descripci\'{o}n del error}}}
		\\ \hline
		No se encontr\'{o} ning\'{u}n error.\\ \hline
	\end{tabularx}
\end{table}
\newpage
\textbf{Comentarios y observaciones:}
\begin{table}[H]
	\centering
	\setlength{\extrarowheight}{\altocelda}
	\begin{tabularx}{\anchotabla}{|X|}
		\hline
		Se verific\'{o} la funcionalidad proporcionando una contrase\~{n}a incorrecta en las operaciones de inicio de sesi\'{o}n y modificaci\'{o}n, en cuyo caso el software genera un mensaje de error al usuario.\\ \hline
	\end{tabularx}
\end{table}

\begin{minipage}[t]{0.45\textwidth}
	\begin{flushleft}
		\textbf{Elaborado por:} \nombre
	\end{flushleft}
\end{minipage}
\begin{minipage}[t]{0.45\textwidth}
	\begin{flushright}
		\begin{center}
			\textbf{Fecha:} \fecha
		\end{center}
	\end{flushright}
\end{minipage}
\vfill
\newpage
\begin{center}
	\textbf{Gui\'{o}n de prueba 4}
\end{center}

\textbf{Nombre del proyecto:} \proyecto

\textbf{\'{A}rea funcional:} Historias.

\textbf{Nombre de la prueba:} Gestionar historias.
\vfill
\textbf{Prop\'{o}sito:}
\begin{table}[h]
	\centering
	\setlength{\extrarowheight}{\altocelda}
	\begin{tabularx}{\anchotabla}{|X|}
		\hline
		Probar la funcionalidad que implica gestionar las historias m\'{e}dicas.\\ \hline
	\end{tabularx}
\end{table}

\textbf{Resultado:}
\begin{table}[h]
	\centering
	\setlength{\extrarowheight}{\altocelda}
	\begin{tabularx}{\anchotabla}{|X|}
		\hline
		Prueba superada con \'{e}xito.\\ \hline
	\end{tabularx}
\end{table}

\begin{table}[h]
		\centering
		\setlength{\extrarowheight}{\altocelda}
		\begin{tabulary}{\anchotabla}{|c|J|J|J|}
			\hline
			\thead{\textbf{\small{\#}}} & \thead{\textbf{\small{Acci\'{o}n}}} & \thead{\textbf{\small{Opci\'{o}n}}} & \thead{\textbf{\small{Resultado esperado}}}\\ \hline

			1 & El usuario registra una historia, suministrando datos personales del paciente y su contrase\~{n}a de usuario. & Se selecciona la opci\'{o}n <<Registrar historia>>, ubicada en el men\'{u} historia. & Ventana de confirmaci\'{o}n, informando al usuario que ha registrado la historia correctamente.\\ \hline
		
			2 & El usuario busca una historia existente. & Se selecciona la opci\'{o}n <<Buscar historia>>, ubicada en el men\'{u} historia. & Ventana de confirmaci\'{o}n, informando al usuario que ha abierto la historia correctamente.\\ \hline
			
			3 & El usuario ve los datos de la historia. & Se selecciona la opci\'{o}n <<Ver historia>>, ubicada en el men\'{u} historia. & Ventana de muestra los datos de la historia m\'{e}dica.\\ \hline
			
			4 & El usuario modifica los datos de la historia, proporcionando su contrase\~{n}a de usuario. & Se selecciona la opci\'{o}n <<Modificar historia>>, ubicada en el men\'{u} historia. & Ventana de confirmaci\'{o}n, informando al usuario que ha modificado la historia correctamente.\\ \hline
			
			5 & El usuario cierra la historia. & Se selecciona la opci\'{o}n <<Cerrar historia>>, ubicada en el men\'{u} historia. & Ventana de confirmaci\'{o}n, informando al usuario que ha cerrado la historia correctamente.\\ \hline
			
			6 & El usuario elimina la historia, proporcionando su contrase\~{n}a de usuario. & Se selecciona la opci\'{o}n <<Eliminar historia>>, ubicada en el men\'{u} historia. & Ventana de confirmaci\'{o}n, informando al usuario que ha eliminado la historia correctamente.\\ \hline
		\end{tabulary}
\end{table}

\textbf{Errores encontrados:}
\begin{table}[H]
	\centering
	\setlength{\extrarowheight}{\altocelda}
	\begin{tabularx}{\anchotabla}{|X|}
		\hline
		\thead{\textbf{\small{Descripci\'{o}n del error}}}
		\\ \hline
		No se encontr\'{o} ning\'{u}n error.\\ \hline
	\end{tabularx}
\end{table}

\textbf{Comentarios y observaciones:}
\begin{table}[H]
	\centering
	\setlength{\extrarowheight}{\altocelda}
	\begin{tabularx}{\anchotabla}{|X|}
		\hline
		Se verific\'{o} la funcionalidad proporcionando una contrase\~{n}a incorrecta en las operaciones de registro, modificaci\'{o}n y eliminaci\'{o}n, en cuyo caso el software genera un mensaje de error al usuario.\\ \hline
	\end{tabularx}
\end{table}

\begin{minipage}[t]{0.45\textwidth}
	\begin{flushleft}
		\textbf{Elaborado por:} \nombre
	\end{flushleft}
\end{minipage}
\begin{minipage}[t]{0.45\textwidth}
	\begin{flushright}
		\begin{center}
			\textbf{Fecha:} \fecha
		\end{center}
	\end{flushright}
\end{minipage}
\vfill
\newpage
\begin{center}
	\textbf{Gui\'{o}n de prueba 5}
\end{center}

\textbf{Nombre del proyecto:} \proyecto

\textbf{\'{A}rea funcional:} Muestras.

\textbf{Nombre de la prueba:} Gestionar muestras.
\vfill
\textbf{Prop\'{o}sito:}
\begin{table}[h]
	\centering
	\setlength{\extrarowheight}{\altocelda}
	\begin{tabularx}{\anchotabla}{|X|}
		\hline
		Probar la funcionalidad que implica gestionar muestras.\\ \hline
	\end{tabularx}
\end{table}

\textbf{Resultado:}
\begin{table}[h]
	\centering
	\setlength{\extrarowheight}{\altocelda}
	\begin{tabularx}{\anchotabla}{|X|}
		\hline
		Prueba superada con \'{e}xito.\\ \hline
	\end{tabularx}
\end{table}

\begin{table}[h]
		\centering
		\setlength{\extrarowheight}{\altocelda}
		\begin{tabulary}{\anchotabla}{|c|J|J|J|}
			\hline
			\thead{\textbf{\small{\#}}} & \thead{\textbf{\small{Acci\'{o}n}}} & \thead{\textbf{\small{Opci\'{o}n}}} & \thead{\textbf{\small{Resultado esperado}}}\\ \hline

			1 & El usuario registra una muestra, suministrando los datos de la misma y su contrase\~{n}a de usuario. & Se selecciona la opci\'{o}n <<Registrar muestra>>, ubicada en el men\'{u} muestra. & Ventana de confirmaci\'{o}n, informando al usuario que ha registrado la muestra correctamente.\\ \hline
		
			2 & El usuario busca una muestra existente. & Se selecciona la opci\'{o}n <<Buscar muestra>>, ubicada en el men\'{u} muestra. & Ventana de confirmaci\'{o}n, informando al usuario que ha abierto la muestra correctamente.\\ \hline
			
			3 & El usuario ve los datos de la muestra. & Se selecciona la opci\'{o}n <<Ver muestra>>, ubicada en el men\'{u} muestra. & Ventana que proporciona los datos de la muestra.\\ \hline
			
			4 & El usuario exporta los datos de la muestra. & Se selecciona la opci\'{o}n <<Exportar muestra>>, ubicada en el men\'{u} muestra. & Ventana de confirmaci\'{o}n, informando al usuario que ha exportado la muestra correctamente.\\ \hline
			
			5 & El usuario modifica los datos de la muestra, proporcionando su contrase\~{n}a de usuario. & Se selecciona la opci\'{o}n <<Modificar muestra>>, ubicada en el men\'{u} muestra. & Ventana de confirmaci\'{o}n, informando al usuario que ha modificado la muestra correctamente.\\ \hline
			
			6 & El usuario cierra la muestra. & Se selecciona la opci\'{o}n <<Cerrar muestra>>, ubicada en el men\'{u} muestra. & Ventana de confirmaci\'{o}n, informando al usuario que ha cerrado la muestra correctamente.\\ \hline
			
			7 & El usuario elimina la muestra, proporcionando su contrase\~{n}a de usuario. & Se selecciona la opci\'{o}n <<Eliminar muestra>>, ubicada en el men\'{u} muestra. & Ventana de confirmaci\'{o}n, informando al usuario que ha eliminado la muestra correctamente.\\ \hline
		\end{tabulary}
\end{table}
\newpage
\textbf{Errores encontrados:}
\begin{table}[H]
	\centering
	\setlength{\extrarowheight}{\altocelda}
	\begin{tabularx}{\anchotabla}{|X|}
		\hline
		\thead{\textbf{\small{Descripci\'{o}n del error}}}
		\\ \hline
		No se encontr\'{o} ning\'{u}n error.\\ \hline
	\end{tabularx}
\end{table}

\textbf{Comentarios y observaciones:}
\begin{table}[H]
	\centering
	\setlength{\extrarowheight}{\altocelda}
	\begin{tabularx}{\anchotabla}{|X|}
		\hline
		Se verific\'{o} la funcionalidad proporcionando una contrase\~{n}a incorrecta en las operaciones de registro, modificaci\'{o}n y eliminaci\'{o}n, en cuyo caso el software genera un mensaje de error al usuario.\\ \hline
	\end{tabularx}
\end{table}

\begin{minipage}[t]{0.45\textwidth}
	\begin{flushleft}
		\textbf{Elaborado por:} \nombre
	\end{flushleft}
\end{minipage}
\begin{minipage}[t]{0.45\textwidth}
	\begin{flushright}
		\begin{center}
			\textbf{Fecha:} \fecha
		\end{center}
	\end{flushright}
\end{minipage}
\vfill
\newpage
\begin{center}
	\textbf{Gui\'{o}n de prueba 6}
\end{center}

\textbf{Nombre del proyecto:} \proyecto

\textbf{\'{A}rea funcional:} Administraci\'{o}n de usuarios.

\textbf{Nombre de la prueba:} Gestionar usuarios.
\vfill
\textbf{Prop\'{o}sito:}
\begin{table}[h]
	\centering
	\setlength{\extrarowheight}{\altocelda}
	\begin{tabularx}{\anchotabla}{|X|}
		\hline
		Probar la funcionalidad que implica gestionar usuarios.\\ \hline
	\end{tabularx}
\end{table}

\textbf{Resultado:}
\begin{table}[h]
	\centering
	\setlength{\extrarowheight}{\altocelda}
	\begin{tabularx}{\anchotabla}{|X|}
		\hline
		Prueba superada con \'{e}xito.\\ \hline
	\end{tabularx}
\end{table}

\begin{table}[h]
		\centering
		\setlength{\extrarowheight}{\altocelda}
		\begin{tabulary}{\anchotabla}{|c|J|J|J|}
			\hline
			\thead{\textbf{\small{\#}}} & \thead{\textbf{\small{Acci\'{o}n}}} & \thead{\textbf{\small{Opci\'{o}n}}} & \thead{\textbf{\small{Resultado esperado}}}\\ \hline

			1 & El administrador registra un usuario, suministrando los datos personales del mismo y su contrase\~{n}a de administrador. & Se selecciona la opci\'{o}n <<Registrar usuario>>, ubicada en el men\'{u} usuario. & Ventana de confirmaci\'{o}n, informando al administrador que ha registrado al nuevo usuario correctamente.\\ \hline
		
			2 & El administrador gestiona un usuario existente. & Se selecciona la opci\'{o}n <<Administrar usuarios>>, ubicada en el men\'{u} usuario. & Ventana de administraci\'{o}n de usuarios.\\ \hline
			
			3 & El administrador ve los datos de un usuario. & Se selecciona la opci\'{o}n <<Ver usuario>>, ubicada en la ventana administrar usuarios. & Ventana que muestra los datos del usuario.\\ \hline
			
			4 & El administrador cambia el rol de un usuario, proporcionando su contrase\~{n}a de administrador. & Se selecciona la opci\'{o}n <<Cambiar rol>>, ubicada en la ventana administrar usuarios. & Ventana de confirmaci\'{o}n, informando al administrador que ha cambiado el rol del usuario correctamente.\\ \hline
			
			5 & El administrador cambia la contrase\~{n}a de un usuario, proporcionando su contrase\~{n}a de administrador. & Se selecciona la opci\'{o}n <<Cambiar contrase\~{n}a>>, ubicada en la ventana administrar usuario. & Ventana de confirmaci\'{o}n, informando al administrador que ha cambiado la contrase\~{n}a del usuario correctamente.\\ \hline
			
			6 & El administrador elimina un usuario, proporcionando su contrase\~{n}a de administrador. & Se selecciona la opci\'{o}n <<Eliminar usuario>>, ubicada en la ventana administrar usuarios. & Ventana de confirmaci\'{o}n, informando al administrador que ha eliminado al usuario correctamente.\\ \hline
		\end{tabulary}
\end{table}
\newpage
\textbf{Errores encontrados:}
\begin{table}[H]
	\centering
	\setlength{\extrarowheight}{\altocelda}
	\begin{tabularx}{\anchotabla}{|X|}
		\hline
		\thead{\textbf{\small{Descripci\'{o}n del error}}}
		\\ \hline
		No se encontr\'{o} ning\'{u}n error.\\ \hline
	\end{tabularx}
\end{table}

\textbf{Comentarios y observaciones:}
\begin{table}[H]
	\centering
	\setlength{\extrarowheight}{\altocelda}
	\begin{tabularx}{\anchotabla}{|X|}
		\hline
		Se verific\'{o} la funcionalidad proporcionando una contrase\~{n}a incorrecta en las operaciones de cambio de rol, cambio de contrase\~{n}a y eliminaci\'{o}n, en cuyo caso el software genera un mensaje de error al administrador.\\ \hline
	\end{tabularx}
\end{table}

\begin{minipage}[t]{0.45\textwidth}
	\begin{flushleft}
		\textbf{Elaborado por:} \nombre
	\end{flushleft}
\end{minipage}
\begin{minipage}[t]{0.45\textwidth}
	\begin{flushright}
		\begin{center}
			\textbf{Fecha:} \fecha
		\end{center}
	\end{flushright}
\end{minipage}
\vfill
\newpage
%%%%%%%%%%%%%%%%%%%%%%%%%%%%%%%%%%%%%%%%%%%%%%%%%%%%%%%%%%%%%%%%%%%%%%
%%%%%%%%%%%%%%%%%%%%%%%%%%%%%%%%%%%%%%%%%%%%%%%%%%%%%%%%%%%%%%%%%%%%%%
\begin{center}
	\textbf{Prueba de aceptaci\'{o}n}
\end{center}

\textbf{Nombre del proyecto:} \proyecto

\textbf{Usuario:} Dermat\'{o}logo.

\textbf{Prop\'{o}sito:}
\begin{table}[h]
	\centering
	\setlength{\extrarowheight}{\altocelda}
	\begin{tabularx}{\anchotabla}{|X|}
		\hline
		Evaluar el cumplimiento y la satisfacci\'{o}n de los requerimientos establecidos durante la definici\'{o}n del proyecto.\\ \hline
	\end{tabularx}
\end{table}

\textbf{Resultado:}

\begin{table}[h]
	\centering
	\setlength{\extrarowheight}{\altocelda}
	\begin{tabulary}{\anchotabla}{|c|L|J|c|}
		\hline
		\thead{\textbf{\small{\#}}} & \thead{\textbf{\small{Requerimiento}}} & \thead{\textbf{\small{Acci\'{o}n}}} & \thead{\textbf{\small{Resultado}}}\\ \hline

			1 & Conectar el \hbox{MiniScan}. & Conectar el software con el MiniScan. & \\ \hline
		
			2 & Desconectar el \hbox{MiniScan}. & Desconectar el software del MiniScan. & \\ \hline
			
			3 & Calibrar el \hbox{MiniScan}. & Calibrar el MiniScan, utilizando una trampa de luz y una cer\'{a}mica blanca. & \\ \hline
			
			4 & Realizar una medici\'{o}n. & Realizar una medici\'{o}n con el MiniScan. & \\ \hline
			
			5 & Consultar resultados. & Consultar los resultados de una medici\'{o}n realizada. & \\ \hline
			
			6 & Borrar resultados. & Borrar los resultados obtenidos de una medici\'{o}n realizada. & \\ \hline
			
			7 & Iniciar sesi\'{o}n. & Iniciar la sesi\'{o}n de usuario. & \\ \hline
			
			8 & Consultar usuario. & Consultar los datos personales del usuario. & \\ \hline
			
			9 & Modificar usuario. & Modificar los datos personales del usuario. & \\ \hline
			
			10 & Cerrar sesi\'{o}n. & Cerrar la sesi\'{o}n de usuario. & \\ \hline
			
			11 & Registrar historia. & Registrar la historia m\'{e}dica de un paciente. & \\ \hline
			
			12 & Consultar historia. & Consultar los datos de la historia m\'{e}dica de un paciente. & \\ \hline
			
			13 & Modificar historia. & Modificar los datos de la historia m\'{e}dica de un paciente. & \\ \hline
			
			14 & Cerrar historia. & Cerrar la historia m\'{e}dica de un paciente. & \\ \hline
			
			15 & Eliminar historia. & Eliminar la historia m\'{e}dica de un paciente. & \\ \hline
			
			16 & Registrar muestra. & Registrar una muestra perteneciente a una historia. & \\ \hline
			
			17 & Consultar muestra. & Consultar los datos de una muestra perteneciente a una historia. & \\ \hline
			
			18 & Exportar muestra. & Exportar los datos de una muestra a un archivo port\'{a}til. & \\ \hline

	\end{tabulary}
\end{table}

\begin{table}[h]
		\centering
		\setlength{\extrarowheight}{\altocelda}
		\begin{tabulary}{\anchotabla}{|c|L|J|c|}
			\hline
			\thead{\textbf{\small{\#}}} & \thead{\textbf{\small{Requerimiento}}} & \thead{\textbf{\small{Acci\'{o}n}}} & \thead{\textbf{\small{Resultado}}}\\ \hline
	
			19 & Modificar muestra. & Modificar los datos de una muestra perteneciente a una historia. & \\ \hline
			
			20 & Cerrar muestra. & Cerrar una muestra perteneciente a una historia. & \\ \hline
			
			21 & Eliminar muestra. & Eliminar una muestra perteneciente a una historia. & \\ \hline
	
	\end{tabulary}
\end{table}

\FloatBarrier
\textbf{Comentarios y observaciones:}
\begin{table}[H]
	\centering
	\setlength{\extrarowheight}{\altocelda}
	\begin{tabularx}{\anchotabla}{|X|}
		\hline
		\null
		\null
		\null		
		\\ \hline
	\end{tabularx}
\end{table}

\begin{minipage}[t]{0.45\textwidth}
	\begin{flushleft}
		\textbf{Elaborado por:}
	\end{flushleft}
\end{minipage}
\begin{minipage}[t]{0.45\textwidth}
	\begin{flushright}
		\begin{center}
			\textbf{Fecha:}
		\end{center}
	\end{flushright}
\end{minipage}
\vfill

\newpage
\begin{center}
	\textbf{Constancia de prueba de aceptaci\'{o}n}
\end{center}

Por medio de la presente, se hace constar que el cliente Sandra Vivas, luego de una reuni\'{o}n realizada el 20 de octubre de 2015, donde se presentaron y probaron las diferentes funcionalidades del software Spectrasoft, ha manifestado su satisfacci\'{o}n y ha aceptado su producto. En virtud de lo cual, firma la presente constancia de aceptaci\'{o}n, junto con el desarrollador.

\null
\null
\null

\begin{minipage}[t]{0.45\textwidth}
	\begin{flushleft}
		\begin{center}
			\textbf{Cliente}
		\end{center}
	\end{flushleft}
\end{minipage}
\begin{minipage}[t]{0.45\textwidth}
	\begin{flushright}
		\begin{center}
			\textbf{Desarrollador}
		\end{center}
	\end{flushright}
\end{minipage}

\newpage
\begin{center}
	\textbf{Prueba de aceptaci\'{o}n}
\end{center}

\textbf{Nombre del proyecto:} \proyecto

\textbf{Usuario:} Investigador.

\textbf{Prop\'{o}sito:}
\begin{table}[h]
	\centering
	\setlength{\extrarowheight}{\altocelda}
	\begin{tabularx}{\anchotabla}{|X|}
		\hline
		Evaluar el cumplimiento y la satisfacci\'{o}n de los requerimientos establecidos durante la definici\'{o}n del proyecto.\\ \hline
	\end{tabularx}
\end{table}

\textbf{Resultado:}

\begin{table}[h]
	\centering
	\setlength{\extrarowheight}{\altocelda}
	\begin{tabulary}{\anchotabla}{|c|L|J|c|}
		\hline
		\thead{\textbf{\small{\#}}} & \thead{\textbf{\small{Requerimiento}}} & \thead{\textbf{\small{Acci\'{o}n}}} & \thead{\textbf{\small{Resultado}}}\\ \hline

			1 & Conectar el \hbox{MiniScan}. & Conectar el software con el MiniScan. & \\ \hline
		
			2 & Desconectar el \hbox{MiniScan}. & Desconectar el software del MiniScan. & \\ \hline
			
			3 & Calibrar el \hbox{MiniScan}. & Calibrar el MiniScan, utilizando una trampa de luz y una cer\'{a}mica blanca. & \\ \hline
			
			4 & Realizar una medici\'{o}n. & Realizar una medici\'{o}n con el MiniScan. & \\ \hline
			
			5 & Consultar resultados. & Consultar los resultados de una medici\'{o}n realizada. & \\ \hline
			
			6 & Borrar resultados. & Borrar los resultados obtenidos de una medici\'{o}n realizada. & \\ \hline
			
			7 & Iniciar sesi\'{o}n. & Iniciar la sesi\'{o}n de usuario. & \\ \hline
			
			8 & Consultar usuario. & Consultar los datos personales del usuario. & \\ \hline
			
			9 & Modificar usuario. & Modificar los datos personales del usuario. & \\ \hline
			
			10 & Cerrar sesi\'{o}n. & Cerrar la sesi\'{o}n de usuario. & \\ \hline
			
			11 & Consultar historia. & Consultar los datos de la historia m\'{e}dica de un paciente. & \\ \hline
			
			12 & Cerrar historia. & Cerrar la historia m\'{e}dica de un paciente. & \\ \hline
			
			13 & Consultar muestra. & Consultar los datos de una muestra perteneciente a una historia. & \\ \hline
			
			14 & Exportar muestra. & Exportar los datos de una muestra a un archivo port\'{a}til. & \\ \hline
			
			15 & Cerrar muestra. & Cerrar una muestra perteneciente a una historia. & \\ \hline

	\end{tabulary}
\end{table}
\newpage
\FloatBarrier
\textbf{Comentarios y observaciones:}
\begin{table}[H]
	\centering
	\setlength{\extrarowheight}{\altocelda}
	\begin{tabularx}{\anchotabla}{|X|}
		\hline
		\null
		\null
		\null
		\\ \hline
	\end{tabularx}
\end{table}

\begin{minipage}[t]{0.45\textwidth}
	\begin{flushleft}
		\textbf{Elaborado por:}
	\end{flushleft}
\end{minipage}
\begin{minipage}[t]{0.45\textwidth}
	\begin{flushright}
		\begin{center}
			\textbf{Fecha:}
		\end{center}
	\end{flushright}
\end{minipage}
\vfill

\newpage
\begin{center}
	\textbf{Constancia de prueba de aceptaci\'{o}n}
\end{center}

Por medio de la presente, se hace constar que el cliente Aar\'{o}n Mu\~{n}oz, luego de una reuni\'{o}n realizada el 20 de octubre de 2015, donde se presentaron y probaron las diferentes funcionalidades del software Spectrasoft, ha manifestado su satisfacci\'{o}n y ha aceptado su producto. En virtud de lo cual, firma la presente constancia de aceptaci\'{o}n, junto con el desarrollador.

\null
\null
\null

\begin{minipage}[t]{0.45\textwidth}
	\begin{flushleft}
		\begin{center}
			\textbf{Cliente}
		\end{center}
	\end{flushleft}
\end{minipage}
\begin{minipage}[t]{0.45\textwidth}
	\begin{flushright}
		\begin{center}
			\textbf{Desarrollador}
		\end{center}
	\end{flushright}
\end{minipage}

\newpage
\begin{center}
	\textbf{Constancia de prueba de aceptaci\'{o}n}
\end{center}

Por medio de la presente, se hace constar que el cliente Freddy Narea, luego de una reuni\'{o}n realizada el 20 de octubre de 2015, donde se presentaron y probaron las diferentes funcionalidades del software Spectrasoft, ha manifestado su satisfacci\'{o}n y ha aceptado su producto. En virtud de lo cual, firma la presente constancia de aceptaci\'{o}n, junto con el desarrollador.

\null
\null
\null

\begin{minipage}[t]{0.45\textwidth}
	\begin{flushleft}
		\begin{center}
			\textbf{Cliente}
		\end{center}
	\end{flushleft}
\end{minipage}
\begin{minipage}[t]{0.45\textwidth}
	\begin{flushright}
		\begin{center}
			\textbf{Desarrollador}
		\end{center}
	\end{flushright}
\end{minipage}
\newpage
%%%%%%%%%%%%%%%%%%%%%%%%%%%%%%%%%%%%%%%%%%%%%%%%%%%%%%%%%%%%%%%%%%%%%%
%%%%%%%%%%%%%%%%%%%%%%%%%%%%%%%%%%%%%%%%%%%%%%%%%%%%%%%%%%%%%%%%%%%%%%
\begin{center}
	\textbf{Prueba de evaluaci\'{o}n heur\'{i}stica para la usabilidad del software}
\end{center}

\textbf{Facilidad para entender el software}
\begin{table}[!h]
\begin{center}
\setlength{\extrarowheight}{\altocelda}
	\begin{tabulary}{\anchotabla}{| c | J | c | c | c |}
\hline
\multirow{2}{*}{\textbf{\'{I}tem}} & \multirow{2}{0.6\textwidth}{\centering \textbf{\small{Aspecto}}} & \multicolumn{3}{ c |}{\textbf{Puntuaci\'{o}n}}  \\ 
\cline{3-5}
& & \textbf{0} & \textbf{0.5} & \textbf{1} \\
\hline

1 & Hace hincapi\'{e} en las tareas de mayor prioridad, de forma que los usuarios tengan un punto de referencia claro en la vista principal. &  &  &  \\ \hline
2 & Muestra el nombre y/o logotipo del software en una ubicaci\'{o}n relevante. &  &   &  \\ \hline
3 & La vista principal se diferencia claramente de todas las dem\'{a}s vistas del software. &   &  & \\ \hline
4 & Proporciona un mecanismo de <<informaci\'{o}n>> o <<cont\'{a}ctenos>>, el cual especifica la finalidad del software. &  &  & \\ \hline
5 & Se utiliza un lenguaje dirigido al usuario. Etiqueta las secciones y las categor\'{i}as en funci\'{o}n al valor que tienen para el cliente y no en funci\'{o}n del servicio que prestan dentro de la instituci\'{o}n. &  &   & \\ \hline
6 & Utiliza correctamente las may\'{u}sculas y otras reglas de estilo. &  &  & \\ \hline
7 & Se utiliza un lenguaje imperativo como <<Introduzca una ciudad o c\'{o}digo postal>> s\'{o}lo para las tareas obligatorias. &  &   &  \\ \hline
8 & Los signos de exclamaci\'{o}n se utilizan muy poco. &  &   &  \\ \hline
9 & No aparecen instrucciones gen\'{e}ricas, como <<haga click aqu\'{i}>>, como nombre de una opci\'{o}n. &  &   &  \\ \hline

10 & El t\'{i}tulo de la ventanas empieza por una palabra que transmite informaci\'{o}n . &  &   &  \\ \hline

11 & Los t\'{i}tulos de las ventanas tienen entre 6 y 8 palabras, o menos de 64 caracteres. &  &   &  \\ \hline

12 & La respuesta del software es predecible antes de hacer click sobre una opci\'{o}n. &  &   &  \\ \hline

13 & Cuando se produce un error, se informa de forma clara y no alarmista al usuario de lo ocurrido. &  &   &  \\ \hline

\multicolumn{2}{|r|}{Sub-total:} &  &  &  \\ \hline

\end{tabulary}
\end{center}
\end{table}

\newpage

\textbf{Facilidad para operar el software}
\begin{table}[!h]
\begin{center}
\setlength{\extrarowheight}{\altocelda}
	\begin{tabulary}{\anchotabla}{| c | J | c | c | c |}
\hline
\multirow{2}{*}{\textbf{\'{I}tem}} & \multirow{2}{0.6\textwidth}{\centering \textbf{\small{Aspecto}}} & \multicolumn{3}{ c |}{\textbf{Puntuaci\'{o}n}}  \\ 
\cline{3-5}
& & \textbf{0} & \textbf{0.5} & \textbf{1} \\
\hline

14 & Las opciones est\'{a}n agrupadas de tal forma que las opciones semejantes quedan juntas. &  &  &  \\ \hline

15 & Se utilizan solamente iconos que ayudan a los usuarios a reconocer una clase de elementos de forma inmediata. &  &  &  \\ \hline

16 & El \'{a}rea de navegaci\'{o}n principal est\'{a} situada en un lugar relevante, preferiblemente junto a la vista principal del software. &  &  &  \\ \hline

17 & No incluye m\'{u}ltiples \'{a}reas de navegaci\'{o}n para los mismos tipos de opciones. &  &  &  \\ \hline

18 & Puede habilitar b\'{u}squedas sencillas, con la posibilidad de b\'{u}squedas avanzadas. &  &  &  \\ \hline

19 & Se ofrece a los usuarios un acceso directo a tareas de alta prioridad en el software. &  &  &  \\ \hline

\multicolumn{2}{|r|}{Sub-total:} &  &  &  \\ \hline

\end{tabulary}
\end{center}
\end{table}

\textbf{Dise\~{n}o atractivo}
\begin{table}[!h]
\begin{center}
\setlength{\extrarowheight}{\altocelda}
	\begin{tabulary}{\anchotabla}{| c | J | c | c | c |}
\hline
\multirow{2}{*}{\textbf{\'{I}tem}} & \multirow{2}{0.6\textwidth}{\centering \textbf{\small{Aspecto}}} & \multicolumn{3}{ c |}{\textbf{Puntuaci\'{o}n}}  \\ 
\cline{3-5}
& & \textbf{0} & \textbf{0.5} & \textbf{1} \\
\hline

20 & Los estilos de fuente y otros formatos de texto, como los tama\~{n}os, los colores, etc., en el software, son sencillos. &  &  &  \\ \hline

21 & Los elementos m\'{a}s importantes de la vista principal son visibles <<en su totalidad>>, sin necesidad de realizar un desplazamiento.  &  &  &  \\ \hline

22 & Se utilizan pocos logotipos. &  &  &  \\ \hline

23 & Se utilizan las gr\'{a}ficas e im\'{a}genes para mostrar el contenido real y no s\'{o}lo para decorar el software. &  &  &  \\ \hline

24 & Se etiquetan las gr\'{a}ficas e im\'{a}genes al no quedar claras en el contexto que las acompa\~{n}a. &  &  &  \\ \hline

25 & Se muestran las gr\'{a}ficas e im\'{a}genes ajust\'{a}ndolas al tama\~{n}o adecuado para su visualizaci\'{o}n. &  &  &  \\ \hline

26 & La imagen del logotipo del software se muestra de forma est\'{a}tica. &  &  &  \\ \hline

27 & Se tienen pocas animaciones. &  &  &  \\ \hline

28 & Se evita utilizar im\'{a}genes de fondo para colocarles texto encima. &  &  &  \\ \hline

29 & Se mantiene una coherencia y uniformidad en las estructuras y colores de todas las vistas del software. &  &  &  \\ \hline

30 & La interfaz del software es limpia, sin causar ruido visual. &  &  &  \\ \hline

31 & Se hace un uso correcto del espacio visual de la interfaz del software. &  &  &  \\ \hline

\multicolumn{2}{|r|}{Sub-total:} &  &  &  \\ \hline

\end{tabulary}
\end{center}
\end{table}
\vfill
\begin{table}[h]
		\centering
		\setlength{\extrarowheight}{\altocelda}
		\begin{tabulary}{\anchotabla}{|r|c|}
			\hline
			
			& \multicolumn{1}{r|}{\textbf{Resultados}} \\ \hline		
			
			\textbf{Total:} & \\ \hline
			
			\textbf{Cantidad de \'{i}tems aplicables:} &  \\ \hline
			
			\textbf{Total dividido entre \'{i}tems:} & \\ \hline
			
			\textbf{Resultado multiplicado por 100:} & \\ \hline

		\end{tabulary}
\end{table}
\newpage
\null
\vfill
Escala [100\%, 90\%] $\rightarrow$ excelente
			
Escala (90\%, 80\%] $\rightarrow$ bueno
			
Escala (80\%, 50\%] $\rightarrow$ deficiente
			
Escala (50\%, 0\%] $\rightarrow$ malo

\textbf{Resultado:}

\vfill

\textbf{Comentarios y observaciones:}
\begin{table}[H]
	\centering
	\setlength{\extrarowheight}{\altocelda}
	\begin{tabularx}{\anchotabla}{|X|}
		\hline
		\null
		\null
		\null		
		\\ \hline
	\end{tabularx}
\end{table}

\begin{minipage}[t]{0.45\textwidth}
	\begin{flushleft}
		\textbf{Elaborado por:}
	\end{flushleft}
\end{minipage}
\begin{minipage}[t]{0.45\textwidth}
	\begin{flushright}
		\begin{center}
			\textbf{Fecha:}
		\end{center}
	\end{flushright}
\end{minipage}

\vfill
	
\end{document}